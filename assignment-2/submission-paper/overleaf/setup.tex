\documentclass[a4paper,12pt]{scrartcl}
\setcounter{secnumdepth}{4}
\setcounter{tocdepth}{4}

\usepackage[nenglish]{babel}
% \addto\captionsngerman{\renewcommand{\refname}{Quellenverzeichnis}}
\usepackage[T1]{fontenc}
\usepackage{lmodern}
\usepackage[utf8]{inputenc}
\usepackage{lipsum} % Erzeugt Blindtext. Kann von euch auskommentiert werden.

% Pakete für Bibtex/Quellenangaben
% Werden bereits hier geladen, weil es sonst zu einer Kollision zwischen zwei Packages kommt.
\usepackage[noadjust]{cite}
\usepackage[hyphens]{url}	% Nutze vorhandene Bindestriche für Zeilenumbrüche in URLs.
\usepackage{etoolbox}
\apptocmd{\sloppy}{\hbadness 10000\relax}{}{}	% Ignoriere Boxen mit zu viel Weißraum im 'sloppy'-Modus.

% Pakete für Einheiten
\usepackage{siunitx}
\sisetup{
  output-decimal-marker = {,},
  inter-unit-product = \ensuremath{{}\cdot{}},
  range-phrase = ...
}

% Pakete für Formeln
\usepackage{amsfonts}
\usepackage{amsmath}
\usepackage{amssymb}
\usepackage{bigints}
\renewcommand{\theequation}{\arabic{section}.\arabic{subsection}}	% Formeln mit genauer Abschnittangabe beschriften.
\numberwithin{equation}{subsection}  % Die Nummerierung einer Gleichung hinten anhängen.
\newcommand\numberthis{\addtocounter{equation}{1}\tag{\theequation}} % Einzelne Nummerierung innerhalb von 'align*'-Umgebung.

% Pakete für Quellcode und Verzeichnisse
% Werden bereits hier geladen, weil es sonst zu einer Kollision zwischen zwei Packages kommt.
\usepackage{enumerate}
\usepackage[]{listofsymbols}
\usepackage{listings}
\renewcommand{\lstlistlistingname}{Codelistenverzeichnis}
\usepackage{scrhack}  % Löst Probleme mit veralteten KOMA-Befehlen im 'listings'-Package.
\usepackage{paralist} % Mehr Optionen beim Erstellen von Aufzählungen ('\enumerate').
%\usepackage[usenames,dvipsnames]{xcolor}
\usepackage[usenames,dvipsnames,table,xcdraw]{xcolor}
\lstset{
	basicstyle = \scriptsize\ttfamily,
  keywordstyle = \bfseries\ttfamily\color{NavyBlue},
  stringstyle = \color{violet}\ttfamily,
  commentstyle = \color{green}\ttfamily,
  emph = {square}, 
  emphstyle = \color{blue}\texttt,
  emph = {[2]root,base},
  emphstyle = {[2]\color{yac}\texttt},
  language = c,
  tabsize = 2,
  basicstyle = \footnotesize\ttfamily,
  numbers = left,
  numberfirstline,
  breaklines = true,
  breakatwhitespace = true,
  linewidth = \textwidth,
  xleftmargin = 0.075\textwidth,
  frame = tlrb,
  captionpos = b,
  inputencoding = {utf8},
  extendedchars = false
}

% Pakete für Grafiken
\usepackage{graphicx, epstopdf}
%\usepackage{subfig}
\usepackage{subcaption}
\usepackage{rotating}
\usepackage{float}
\usepackage{picinpar}
\usepackage[section]{placeins}
%\usepackage{hyperref}					    % Wird ganz zum Schluss geladen (s.u.), sonst kommt es zu Warnungen.
\usepackage{microtype}              % Schriftbildverschönerung
\usepackage{textcomp}               % Fügt zusätzliche Symbole im Textmodus ein.
\usepackage[format = hang]{caption} % Einstellungen für Bildunter- bzw. überschriften:
\pdfminorversion = 6

\usepackage{matlab-prettifier}
% Pakete für Schaltpläne/Zeichnungen
\usepackage{pgfplots}
\pgfplotsset{compat = 1.13,
             /pgf/number format/.cd,
             use comma,
             1000 sep = {}}
\usepackage{tikz}							    % TIKZ-Paket
\usepackage{circuitikz}					  % Schaltpläne mit TIKZ erstellen.
\usetikzlibrary{matrix,           % Mehr Optionen zum Erstellen und Formatieren von Matrizen.
                shapes,           % Stellt weitere geometrische Formen zur Verfügung.
                arrows.meta,      % Mehr Optionen zum Erstellen und Formatieren von Pfeilen.
                positioning,      % Erweiterte Optionen zum Platzieren von Objekten.
                circuits.ee.IEC,  % Beinhaltet die offiziellen IEC-Schaltzeichen.
                calc}             % Ermöglicht komplexe Berechnungen mit Koordinaten.
\usepackage{marvosym}             % Fügt einige Symbole zur Verwendung in Grafiken hinzu.

\newcommand{\sbt}{\,\begin{picture}(-1,1)(-1,-3)\circle*{3}\end{picture}\ }

% Pakete für Tabellen
\usepackage{array}
\usepackage{tabularx}
\newcolumntype{w}[1]{>{\raggedleft\hspace{0pt}}p{#1}}
\usepackage{bigdelim}         % Ermöglicht bessere Formatierung der Zellen untereinander.
\usepackage{booktabs}         % Ermöglicht besseres Tabellen-Layout.

% Pakete für Style/Formatierung
%\usepackage{fancybox}
\usepackage{fancyhdr}
\usepackage{ulem}
\usepackage{setspace}
\usepackage[a4paper,
            lmargin = {25mm},
            rmargin = {25mm},
            tmargin = {25mm},
            bmargin = {25mm}]{geometry}
%\addtolength{\footskip}{-0.8cm}  % Fussbereich 0,8 cm höher, so dass die Seitennummierung höher ist.
\onehalfspacing
%%Kopfzeile erstellen
	\pagestyle{fancy}
	\fancyhf{}
	\fancyhead[R]{
	\begin{tabular}[b]{l}
	\shorttitle %Kopfzeile rechts
	\end{tabular}}
	\fancyhead[L]{
	\begin{tabular}[b]{l}
	\end{tabular}}
	\renewcommand{\headrulewidth}{0.5pt} %obere Trennlinie

%% Fußzeile ersellen
	\fancyfoot[R]{\thepage} %Seitennummer
	\fancyfoot[L]{\shortauthor} %Kopfzeile unten
	\renewcommand{\footrulewidth}{0.4pt} %untere Trennlinie

% Anhang
\usepackage[title, titletoc]{appendix}

%\usepackage{scrlayer-scrpage}				  % KOMA-Paket
\usepackage[pdfpagelabels]{hyperref}	% Muss ganz zum Schluss geladen werden (s.o.).

%\usepackage{graphicx}
%\usepackage[demo]{graphicx}
\usepackage{babel,blindtext}


%datenschutz
\usepackage{censor}

% Sorgt am Anfang eines Absatzes dafür, daß die erste Zeile nicht eingerückt wird.
\setlength{\parindent}{0pt}

% Math unit
\usepackage{siunitx}

% Code packages
\usepackage{algpseudocode}
\usepackage{algorithm}
\usepackage[newfloat]{minted}
\usepackage{xcolor} % to access the named colour LightGray
\definecolor{LightGray}{gray}{0.9}
\newenvironment{code}{\captionsetup{type=listing}}{}
\SetupFloatingEnvironment{listing}{name=Code Listing}


\usepackage{threeparttable}
\renewcommand{\figurename}{Figure}
 \renewcommand{\tablename}{Table}

\usepackage{mathtools}